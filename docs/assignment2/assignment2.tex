%%%%%%%%%%%%%%%%%%%%%%%%%%%%%%%%%%%%%%%%%%%%%%%%%%%%%%%%%%%%%%%%%%%%%%%%%%%%%%%%
%2345678901234567890123456789012345678901234567890123456789012345678901234567890
%        1         2         3         4         5         6         7         8

\documentclass[journal,9pt,onecolumn,draftclsnofoot]{ieeeconf} % Comment this line out if you need a4paper


% See the \addtolength command later in the file to balance the column lengths
% on the last page of the document

% The following packages can be found on http:\\www.ctan.org
\usepackage{graphics} % for pdf, bitmapped graphics files
%\usepackage{epsfig} % for postscript graphics files
%\usepackage{mathptmx} % assumes new font selection scheme installed
%\usepackage{times} % assumes new font selection scheme installed
%\usepackage{amsmath} % assumes amsmath package installed
%\usepackage{amssymb}  % assumes amsmath package installed


%-----------------------------------------------------------------------------
% Graphics 
%-----------------------------------------------------------------------------
\usepackage{graphicx}
%\usepackage{caption}
%\usepackage{subfigure}
%\usepackage{subfloat}

\usepackage[tight,footnotesize]{subfigure}



%-----------------------------------------------------------------------------
% footnote
%-----------------------------------------------------------------------------
\usepackage{footnote}

%-----------------------------------------------------------------------------
% Acronyms  
%-----------------------------------------------------------------------------
\usepackage[nolist]{acronym}


%-----------------------------------------------------------------------------
% In-text author's citation  
%-----------------------------------------------------------------------------


%-----------------------------------------------------------------------------
% Strikethrough  
%-----------------------------------------------------------------------------
\usepackage{soul}

%-----------------------------------------------------------------------------
% Math
%-----------------------------------------------------------------------------
\usepackage{amsmath}
\usepackage{amsfonts}

%-----------------------------------------------------------------------------
% Set the default folder for images
%-----------------------------------------------------------------------------
\graphicspath{{images/}} 

%-----------------------------------------------------------------------------
% References
%-----------------------------------------------------------------------------
\usepackage{hyperref}

%-----------------------------------------------------------------------------
% Algorithms  
%-----------------------------------------------------------------------------
\usepackage{xcolor}
\usepackage[procnames]{listings}
\usepackage{color} 
\definecolor{keywords}{RGB}{255,0,90}
\definecolor{comments}{RGB}{0,0,113}
\definecolor{red}{RGB}{160,0,0}
\definecolor{green}{RGB}{0,150,0}
\definecolor{blue}{rgb}{0.0, 0.0, 0.99}	
\definecolor{backcolour}{rgb}{0.95,0.95,0.92}

\lstset{language=Python, 
        basicstyle=\ttfamily\footnotesize, 
        backgroundcolor=\color{backcolour}, 
        keywordstyle=\color{keywords}\ttfamily,
        commentstyle=\color{comments}\ttfamily,
        stringstyle=\color{red}\ttfamily,
        showstringspaces=false,   
        identifierstyle=\color{green}\ttfamily,
        procnamekeys={def,class}}
        
\makeatletter
\def\BState{\State\hskip-\ALG@thistlm}
\makeatother

\hypersetup{
	colorlinks = true, 
	breaklinks = true, 
	bookmarks = true,
	bookmarksnumbered,
	urlcolor = blue, 
	linkcolor = blue, 
	citecolor = blue, 
	linktoc=page, 
	pdftitle={}, 
	pdfauthor={\textcopyright Author}, 
	pdfsubject={}, 
	pdfkeywords={}, 
	pdfcreator={pdfLaTeX}, % PDF Creator
	pdfproducer={IEEE} 
}

\title{\LARGE \bf
Computer Vision Assignment 2
}

%-----------------------------------------------------------------------------
%\begin{acronym}
 
%\end{acronym}

\author{
 Asem Abdelaziz$^{1}$,
 Mohamed Abdallah  $^{1}$, 
 Taha Ali $^{2}$,
 Abdelrahman Sayed$^{2}$
}

\begin{document}

\maketitle
\thispagestyle{empty}
\pagestyle{empty}


\section*{Problem1: Hough Transform} 
\subsection*{Line Detection} 


The general overview of line detection routine using \textit{Hough Transform}:
\begin{enumerate}

\item Edge detection (eg. using \textit{Canny edge detector}).
\item Mapping each edge point to the \textit{Hough space}.
\item For each mapped point, vote for each candidate line passing through the point, by incrementing an accumulator array at the corresponding line; line is represented by $\theta$ and $\rho$ (i.e angle and distance, respectively).
\item Extract the most candidate lines.

\end{enumerate} 
\subsection*{Implementation}

\textit{Python} is used as a wrapper to perform matrix operations to perform edge detection and lines detection. \textit{OpenCV} is only used for verifying our results, loading/showing images, and drawing lines.


\subsubsection*{Canny Edge Detection}
The following snippet code shows the workflow of \textit{Canny edge detector}, the details of the invoked subroutines are provided in the ~\nameref{appendix:canny-details} 

\begin{lstlisting}
def myCanny( image , ksize , sigma , minThreshold , maxThreshold ) :
    # 1.Guassian Blurring.
    smoothedImage = __cannySmoothing__(image , ksize , sigma)

    # 2.Gradient and Phase Image based on Sobel edge detection.
    gradientImage , phaseImage = __cannyEdgeDetection__(smoothedImage , ksize)

    # 3.Non-Maximum Supression for thinning edges.
    nmsImage = __cannyNonMaximumSupression__(gradientImage , phaseImage)

    # 4.Double Thresholding.
    doubleThresholdedImage = __cannyDoubleThresholding__(nmsImage ,
                                                         minThreshold ,
                                                         maxThreshold)

    # 5.Edge Tracking by Hysterysis.
    edgeTrackedImage = __cannyEdgeTracking__(doubleThresholdedImage ,
                                             maxThreshold ,
                                             minThreshold)

    return edgeTrackedImage
\end{lstlisting}


\subsubsection*{ Hough Lines Detection }

\subsection*{Results}



\section*{Problem2: Active Contour Model}
%----------------------------------------------------------------------------------------
%	BIBLIOGRAPHY
%----------------------------------------------------------------------------------------
\bibliographystyle{IEEEtran}
\bibliography{references.bib} % The file containing the bibliography

\section*{Appendix }
\subsection*{Canny Edge Detector Subroutines}\label{appendix:canny-details}

\include{myCanny}


\end{document}

