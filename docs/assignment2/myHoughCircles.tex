\begin{lstlisting}[caption=Python implementation for Hough Circles Detection , label=imp:hough-circles]
import numpy as np
def myHoughCircles( binaryImage , threshold ) :
    maxRadius = int(np.sqrt(2)*max(binaryImage.shape)  )
    cartesianWidth = binaryImage.shape[ 1 ]
    cartesianHeight = binaryImage.shape[ 0 ]

    accumulator = np.ndarray((cartesianHeight , cartesianWidth , maxRadius) ,
                             dtype = int )
    accumulator.fill(0)

    print "Extracting edge points"
    edge_points = [ (row , col) for row in range(binaryImage.shape[ 0 ]) for col
                    in range(binaryImage.shape[ 1 ]) if
                    binaryImage[ row , col ] != 0 ]

    edge_points = np.asanyarray(edge_points , dtype = np.uint16)

    print "Accumulation"
    for row in range(accumulator.shape[ 0 ]) :
        for col in range(accumulator.shape[ 1 ]) :
            radius = np.round(np.sqrt(np.square(edge_points[ : , 0 ] - row) +
                                      np.square(edge_points[ : , 1 ] - col)))
            radius = radius.astype(dtype = int , copy = False)

            radius , count = np.unique(radius , return_counts = True)
            # count = count.astype(dtype = np.uint16 , copy = False)
            accumulator[ row , col , radius ] += count


    print "Max circle"
    print np.amax(accumulator)
    print "Extract most candidate circles"
    circles = np.where(accumulator > threshold)
    circles = np.transpose(circles)
    circles = tuple(map(tuple , circles))
    circles = [ (col , row , r) for row , col , r in circles ]

    return circles
\end{lstlisting}
